\documentclass[10pt,letterpaper]{article} 
\usepackage[utf8]{inputenc}
\usepackage[left=3cm,right=2cm,top=2cm,bottom=2cm]{geometry}
\usepackage[spanish]{babel}
\usepackage{amsmath}
\usepackage{amsfonts}
\usepackage{amssymb}
\usepackage{graphicx}

\title{Orden de Cambio1}
\author{2021orden-1\\Leticia Blanco Coca}
\date{2021-10-22}

\begin{document}
\maketitle


TIS ha revisado la propuesta que su empresa ha entregado y se ha puntuado de la siguiente manera:

	\begin{table}[h]
		\begin{tabular}{|l|l|r|}
				\hline
                \textbf{Descripcion}                                 & \textbf{\begin{tabular}[c]{@{}l@{}}Puntaje \\ Referencial\end{tabular}} & \multicolumn{1}{l|}{\textbf{\begin{tabular}[c]{@{}l@{}}Puntaje \\ Obtenido\end{tabular}}} \\ \hline
Cumplimiento de especificaciones del proponente & 15 puntos & 4 \\ \hline
Claridad en la organización de la empresa proponente & 10 puntos & 5 \\ \hline
Cumplimiento de especificaciones técnicas & 30 puntos & 6 \\ \hline
Claridad en el proceso de desarrollo & 10 puntos & 7 \\ \hline
Plazo de ejecución & 10 puntos & 8 \\ \hline
Precio total & 15 puntos & 9 \\ \hline
Uso de herramientas en el proceso de desarrollo & 10 puntos & 10 \\ \hline
\end{tabular}

	\end{table}
TIS después de revisar la propuesta de su empresa \textbf{Iridium}, tiene las siguientes observaciones:

	\begin{enumerate}
\item Parte-A, sección AndyJoto, andy tienes que llenar los seeders tambien vago, en su respectivo archivo por favor
\item Parte-B, sección 4.1.1., TIS solicita justificar los montos estipulados como parte de pago de personal, en
cuanto al esfuerzo comprometido y requerido para el desarrollo del proyecto.
\item Parte-B, sección 1.9.1., TIS hace notar que estan planificando en fechas que son feriados y fines de semana,
lo que genera una riesgo de imposibilidad de realizacion. Por lo que, se solicita revisar este apartado.
\item BoletaDeGarantia, sección 4.2., TIS solicita justificar los montos erogados en cada item de los costo de la propuesta
\item CartaDePresentacion, sección Plazo de duración, la vida de la empresa es minima y no genera
confianza a TIS, ya que estas fechas no permiten manteniiento de software

\item Constitucion, sección Previsiones para reservas, que a la letra dice “En caso de fallecimiento, impedimento o incapacidad sobreviniente de uno de los socios, los restantes continuarán con
el giro social, juntamente con los herederos forzosos o legales o los representantes según el caso hasta la
culminación de la gestión anual.”, para fines de este contrato los herederos no forma parte de la sociedad
en ningun contexto. TIS solicita se corrija este apartado.
	\end{enumerate}
Esta adenda de corrección debe ser entregada hasta el \textbf{2021-10-22 a horas 9:30} , en \textbf{ Bloque Informatica UMSS Piso 1}.


Paralelamente se solicita, llenar la planilla adjunta - RESUMENGRUPOEMPRESA; con la información resumen de su propuesta técnica. En este archivo debe registrar el día que su GE ha elegido para el seguimiento de su propuesta de desarrollo en el tiempo que dure el contrato con TIS.


Asímismo, recordar que para el día de la firma del contrato se requiere la entrega de la planilla resumen requerida.
\end{document}
